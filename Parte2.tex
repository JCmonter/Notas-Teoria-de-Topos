\chapter{Gavillas sobre $\Top$}
Dado un espacio $S$, describir sus propiedades globales puede ser una tarea complicada; en muchas ocasiones, resulta conveniente fijarse en sus propiedades locales, es decir, ver qué ocurre en una vecindad alrededor de un punto $x\in S$, por ejemplo. Así, estudiar propiedades globales de un espacio se convierte en un proceso de pasar de local a global. Intuitivamente, esto refleja algo similar a lo que ocurre con una gavilla en un espacio, ya que esta puede entenderse como una asignación $F$ que a cada abierto $U$ en $S$ se le asigna un conjunto $F(U)$, junto con una noción de ``pegado'' entre los abiertos de $S$. Dependiendo de la naturaleza del problema, en lugar de connjuntos, se pueden considerar grupos, anillos, módulos, etc., esto hace que las gavillas sean una herramienta bastante utilizada  en diversas ramas de la matemática, como topología algebraica, la geometría algebraica y, en general, en álgebra. Precisar el origen de la teoría de gavillas resulta complicado, por lo que muchos consideran un punto de partida y motivación para esta teoría el estudio de la \emph{continuación analítica}, introducida por Riemann para la comprensión de las ``funciones multivaluadas''. Aunque, cabe resaltar que, el desarrollo formal de la teoría de gavillas es gracias a Leray, Serre y Cartan. \\
En esta capítulo se presentan los conceptos fundamentales de la teoría de gavillas en conjuntos, para esto nos basamos en las ideas expuestas en \cite{maclane2012sheaves}.
\section{Gavillas sobre $\Top$}
Dado un espacio $S$, denotamos por $\O(S)$ la categoría cuyos objetos son conjuntos abiertos de $S$. Para cada par de abiertos $U$ y $V$ en $\O(S)$, $\rho:U\to V$ morfismo en $\O(S)$ si, y solo si, $U\subseteq V$.
\begin{definition}[Pregavilla]
    Una \emph{pregavilla} en $\O(S)$ es un funtor $F:\O(S)^\op\to\Con$.
\end{definition}
A un elemento $s\in F(U)$ de la imagen de un abierto $U$ de $S$ se le llama \emph{sección}. Al morfismo $F(\rho):F(V)\to F(U)$ correspondiente a la inclusión $\rho:U\hookrightarrow V$ le llamamos \emph{restricción}. Adoptamos la notación $F(\rho)(s)=s\mid_U=\rho^V_U(s)\in F(U)$ con $s$ una sección de $V$. Con esto en mente, una manera explícita en la que podemos entender una pregavilla es la siguiente: para cada par de abiertos $V\subseteq U$ de $S$ el mapeo restricción actúa como $\rho_V^U: F(U)\to F(V)$ tal que para todo $U\in\O(S)$ se  tiene que $\rho_U^U=\mathrm{id}_U$, y para cualesquiera $W\subseteq V\subseteq U\in\O(S)$ se cumple que, $\rho_W^U=\rho_W^V\circ\rho_V^U$.
\begin{lema}
    Sea $F:\O(S)^\op\to\Con$ un funtor, $U$ un abierto de $S$ y $\{U_i\}_{i\in I}$ una cubierta abierta de $U$. Entonces el diagrama 
    \begin{equation}\label{ec:2.1}
        F(U)\xrightarrow{\varepsilon}\prod_{i\in I} F(U_i)\overset{p}{\underset{q}{\rightrightarrows}}\prod_{i,j\in I}F(U_i\cap U_j),
    \end{equation}
    donde para todo $t\in F(U)$, $e(t)=\langle t\mid_{U_i}\rangle_{i\in I}$ y para toda familia $\langle t_i\in F(U_i)\rangle_{i\in I}$, $p(\langle t_i\rangle_{i\in I})=\langle t_i\mid_{(U_i\cap U_j)}\rangle_{i,j\in I}$, siempre existe y se cumple que $pe=qe$.
\end{lema}
\begin{proof}

\end{proof}
\begin{definition}[Pregavilla separada] Una pregavilla $F:\O(S)^{\op}\to\Con$ se dice \emph{separada} si la función $e$ de (\ref{ec:2.1}) es inyectiva. 
\end{definition}
\begin{definition}[Gavilla]
    Una \emph{gavilla} de conjuntos sobre un espacio topológico $S$ es un funtor $F:\O(S)^\op\to\Con$ tal que para todo conjunto abierto $U$ de $S$ existe una cubierta abierta $\{U_i\}_{i\in I}$ de tal modo que (\ref{ec:2.1}) es un igualador.  
\end{definition}
Es decir, una gavilla es pregavilla separada. En la literatura (veáse \cite{tennison1975sheaf}, por ejemplo) es común definir una gavilla como una pregavilla que satisface lo siguiente.
    \begin{enumerate}
        \item[i).] Sea $U\in\O(S)$. Si $\{U_i\}_{i\in I}$ es una cubierta abierta para $U$ y para cualesquiera dos secciones $s$ y $s'$ tales que $\rho^U_{U_i}(s)=\rho^U_{U_i}(s')$, se tiene que $s=s'$.
        \item[ii).] Sea $U\in\O(S)$ y $\{U_i\}_{i\in I}$ una cubierta de $U$. Si $\{s_i\}_{i\in I}$ es una familia de secciones de $F$, donde $s_i\in F(U_i)$, tal que  $\rho^{U_i}_{U_i\cap U_J}(s_i)=\rho^{U_j}_{U_i\cap U_j}$, para toda $i,j\in I$. entonces existe una sección $s\in F(U)$ tal que $\rho^U_{U_i}(s)=s_i$.
    \end{enumerate}
Obsérvese que la sección $s\in F(U)$ definida en ii), es única por la condición i); a esta condición ii) se le conoce comúnmente como condición de pegado. Resulta que en $\Con$, dar una pregavilla separada es equivalente a dar una pregavilla que satisface i) y ii).
\begin{definition}
    Sea $S$ un espacio topológico. La \emph{categoría de gavillas} sobre $S$ consta de lo siguiente.
    \begin{enumerate}
        \item[$\bullet$] \emph{Objetos}. Gavillas de conjuntos sobre $S$.
        \item[$\bullet$] \emph{Morfismos}. Transformaciones naturales.
    \end{enumerate}
\end{definition}
Algunos ejemplos relvantes de gavillas son los siguientes.
\begin{ejp}
\text{}
    \begin{enumerate}
        \item Sea $U\in\O(S)$. El conjunto 
        \begin{eqnarray*}
            C(U):=\{f:U\to\R\mid f\text{ es continua}\}
        \end{eqnarray*}
        es una gavilla. Para esto, primero veamos que $C(-)$ es funtorial, lo cual demostraría que en efecto es una pregavilla.
        Sea $U\overset{\rho}{\hookrightarrow} V$ con $U,V\in\O(S)$. Entonces, $C(\rho):C(V)\to C(U)$. Tomando la restricción usual, se tiene el siguiente diagrama
        \begin{figure}[H]
            \centering
            \includegraphics[width=0.25\linewidth]{img/diagram2.1.png}
        \end{figure}
        conmutativo. Esto prueba la funtorialidad, y así $C(-)$ es una pregavilla.\\
        Ahora, veamos que, para cada $U\in \O(S)$Y $\{U_i\}_{i\in I}$ una cubierta abierta para $U$, el siguiente diagrama 
        \begin{eqnarray}\label{ejp2.1}
            C(U)\xrightarrow{e}\prod_{i\in I} C(U_i)\overset{p}{\underset{q}{\rightrightarrows}} \prod_{i,j\in I}C(U_i\cap U_j)
        \end{eqnarray}
        es un igualador en $\Con$. Sea $\{f_i\}_{i\in I}\in\prod_{i\in I}C(U_i)$ tal que $p(\{f_i\})=q(\{f_i\})$. Entonces, $f_i\mid_{U_i\cap U_j}=f_j\mid_{U_i\cap U_j}$ para toda $i,j\in I$. Para cada $U\in\O(S)$, definimos la función $f:U\to\R$ de la siguiente manera
        \begin{equation*}
            f(x)=\begin{cases}
                f_i(x) & \text{si } x\in Ui,\\
                0 &\text{en otro caso.}
            \end{cases}
        \end{equation*}
        Sea $y\in (a,b)\subset\R$ tal que $f(x)=y$, para algún $x\in U$. Ya que $\{U_i\}_{i\in I}$ es una cubierta abierta para $U$, se tiene que $U=\bigcup_{i\in I}U_i$, y en particular $x\in U_i$, para algún $i\in I$. Así, $f_i(x)=f(x)=y$. Luego, como $\{f_i\}\in C(U_i)$ se sigue que $f_i$ es continua para toda $i\in I$; esto implica que existe un abierto $V\subset U_i\subset U$ tal que $x\in V$ y $f_i(V)=f(V)\subset(a,b)$. Por lo tanto, $f\in C(U)$ y $e(f)=\{f_i\}_{i\in I}$. Ahora, supongamos que existe $g\in CU$ tal que $e(g)=\{f_i\}$. Si $x\in U$ entonces, $x\in U_i$ para alguna $i\in I$. De esto se sigue que, $f(x_i)=f(x)=g(x)$. Al tener que $f$ y $g$ tienen el mismo dominio, mismo codominio y misma regla de correspondencia, así $f=g$; lo cual prueba la unicidad de $f$. Por el Lema \ref{lema:igualador}, se sigue que (\ref{ejp2.1}) es un igualador en $\Con$ y $CU$ es una gavilla. 
        \item Sea $U\in \O(S)$. El conjunto
        \begin{eqnarray*}
            C^\infty(U):=\{f:U\to\R\mid f \text{ es suave}\}
        \end{eqnarray*}
        es una gavilla. La demostración de esto es análoga a la del ejemplo anterior.
        \item Sea $U\in\O(S)$. El conjunto de funciones holomorfas
        \begin{eqnarray*}
            \Omega(U):=\{f:U\to\C\mid f\text{ es holomorfa}\}
        \end{eqnarray*}
        es una gavilla. Primero, veamos que $\Omega(-)$ es una pregavilla. Sea $U\overset{\rho}{\hookrightarrow} V$ con $U,V\in\Omega(S)$. Entonces, 
        \begin{eqnarray*}
            \Omega(\rho):\Omega(V)&\to&\Omega(U).
        \end{eqnarray*}
        Por definición, $\Omega(V)=\{f: V\to\C\mid f\text{ es holomorfa}\}$. Tomando la restricción usual, se tiene que el siguiente diagrama conmuta.
        \begin{figure}[H]
            \centering
            \includegraphics[width=0.2\linewidth]{img/gav3.png}
        \end{figure}
        Así, $\Omega(-)$ es funtorial y, por lo tanto, una pregavilla.\\
        Para probar que $\Omega(-)$ es una gavilla, vamos a verificar las condiciones i) y ii) dadas anteriormente.
        \begin{enumerate}
        \item[i).] Sea $U\in\O(S)$ y $\{U_i\}_{i\in I}$ una cubierta abierta para $U$. Supongamos que $s$ y $s'$ son secciones tales que $\rho^U_{U_i}(s)=\rho^U_{U_i}(s')$. Del siguiente diagrama
            \begin{figure}[H]
            \centering
            \includegraphics[width=0.4\linewidth]{img/diagram2.2.png}
        \end{figure}
        se tiene que $s=s'$.
        \item[ii).] Supongamos que $f_j:U_j\to\mathbb{C}$ son holomorfas y $f_i\mid_{U_i\cap U_j}=f_j\mid_{U_i\cap U_j}$. En $U=\bigcup_{i\in I}U_i$, definimos $f(w)=f_j(w)$, para toda $w\in U_i$. Para ver que la condición de pegado se satisface, basta verificar que $f$ satisface las condiciones de Cauchy-Riemann. Sea $z=x+iy\in\C$. Entonces, $f(z)=f(x,y)=f(x+iy)=u(x,y)+iv(x,y)$. Luego, 
   
        \begin{eqnarray*}
            \dfrac{\partial f}{\partial z}=\dfrac{\partial u}{\partial x}+i\dfrac{\partial v}{\partial y}&=& \dfrac{\partial u'}{\partial x}+i\dfrac{\partial v'}{\partial y}\quad\text{ con }u'(x,y)+iv'(x,y)=f_j(z)\\
            &=& \dfrac{\partial u'}{\partial x}=-\dfrac{\partial v'}{\partial y} \quad f_j\text{ es holomorfa}\\
            &=& \dfrac{\partial u'}{\partial x}_{u_j}=-\dfrac{\partial v'}{\partial y}_{u_j}
        \end{eqnarray*}
        Por lo tanto, $f_j=f\mid_{U_j}$. Así, $f$ es holomorfa, y $\Omega(-)$ es gavilla.
        \end{enumerate}
\end{enumerate}
\end{ejp}
\section{El funtor $\Gamma$} 
Dado $S$ un espacio topológico, la categoría de rebanadas $\Top/S$ consta de lo siguiente.
\begin{enumerate}
    \item[$\bullet$] \emph{Objetos}. Todas las funciones continuas $f\in\Mor(\Top)$ cuyo codominio son igual a $S$.
    \item[$\bullet$] \emph{Morfismos}. Para cada función continua $f:E\to E'\in\Mor(\Top)$, existen $p:E\to S$ y $p':E'\to S$ tales que $p=p'\circ f$.
    \begin{figure}[H]
        \centering
        \includegraphics[width=0.42\linewidth]{img/diagram2.2.1.png}
    \end{figure}
\end{enumerate}
A una función continua de la forma $p:E\to S$ le llamamos \emph{haz} sobre $S$, y al espacio $E$ se le conoce como \emph{espacio total}. En topología algebraica es común estudiar un tipo particular de haces, los cuales se llaman haces fibrados, y no deben de confundirse con los haces que estamos estudiando ya que $f$ solo es una función continua y no un homeomorfismo; además, de que no pedimos la condición de localidad trivial como ocurre en el caso de los haces fibrados. \\
Para cada punto $x\in S$, la \emph{fibra asociada} a $x$ es la preimagen  $p^{-1}(x)$. Esto nos permite interpretar al espacio $E$ como un espacio formado por copias de cada fibra parametrizada por los puntos de $S$. Como $p:E\to S$ es una función continua, la preimagen de abiertos manda abiertos en abiertos; por lo cual, para cada $U\in S$ conjunto abierto, se tiene que $p_U:p^{-1}(U)\to U$ es un haz sobre $U$. Más aún, el siguiente diagrama
    \begin{figure}[H]
        \centering
        \includegraphics[width=0.32\linewidth]{img/diagram2.2.2.png}
    \end{figure}
\noindent
es un pullback en $\Top$. 
\begin{definition}[Sección] Una \emph{sección} (global) de un haz $p:E\to S$ es una función continua $s:S\to E$ tal que $p\circ s=\id_S$. Una \emph{sección local} de $p$ alrededor de $x\in S$ es una sección de $p_U$. 
\end{definition}
\begin{prop}
    Sea $p:E\to S$ un haz y $f:U\to S$ una función continua. Hay una correspondencia biyectiva entre el conjunto de secciones de $p_U$ y el conjunto de levantamientos de $f$ a $E$, es decir, funciones continuas $h:U\to E$ tal que $ph=f$.
\end{prop}
\begin{proof}
    Por la propiedad universal del pullback, dar una sección de $p_U$ es equivalente a dar un levantamiento $h:U\to E$ de $f$ que hace completar el siguiente diagrama.
        \begin{figure}[H]
        \centering
        \includegraphics[width=0.43\linewidth]{img/diagram.2.2.3.png}
    \end{figure}
\end{proof}
\noindent
Denotemos por $\Gamma_pU:=\{s\mid s\text{ es una sección local}\}$. Dado un abierto $U$ de $S$ y $V\subset U$, se define
\begin{eqnarray*}
    \Gamma_pU&\to&\Gamma_p V\\
    s&\mapsto& \rho^U_V(s):=s\mid_V.
\end{eqnarray*}
\begin{lema}
    $\Gamma_p:\O(S)^{op}\to\Con$ es un funtor.
\end{lema}

\section{El funtor $\Lambda$}
Sea $P:\O(S)^{\op}\to\Con$ una pregavilla sobre un espacio $S$. Sea $x\in S$ un punto, y consideremos la familia $\O(x):=\{U\in \O(S)\mid x\in U\}$ la familia de vecindades abiertas de $x$ en $S$. Dados $U_\alpha, U_\beta\in \O(x)$ la intersección $U_\gamma:=U_\alpha\cap U_\beta\in\O(x)$ cumple que $U_\gamma\subseteq U_\alpha$ y $U_\gamma\subseteq U_\beta$. Esto es, considerando el orden pacial $\leq$ dado por la contención de abiertos, se tiene que $(\O(x),\leq)$ es un conjunto dirigido. Además, para cada morfismo  $\iota^\beta_\alpha:U_\alpha\hookrightarrow U_\beta$ en $\O(x)$, se tienen morfismos $\phi^\beta_\alpha:= P(\iota^\beta_\alpha): P(U_\beta)\to P(U_\alpha)$ los cuales forman una familia directa en $\Con$, denotada por $(P(\O(x)),\leq)$. \\
Si el colímite de $(P(\O(x)), \leq )$ en $\Con$ existe, este es denotado por $P_x:=\varinjlim P(U_\alpha)$. Para cada familia $\{M_\alpha\}$ indexada por $\O(x)$, este colímite se puede ver de manera explícita como sigue 
\begin{equation*}
     \varinjlim M_\alpha:=\left(\bigsqcup M_\alpha\right)/\sim
\end{equation*}
donde $s\sim t$ ($s\in PU$ y $t\in PV$) si, y solo si, existe un abierto $W$ tal que $W\subseteq U\cap V$ y  $s\mid_W=t\mid_W\in PW$ para $x\in W$. Además, se tienen morfismos canónicos 
\begin{eqnarray*}
    \phi:P(U_\alpha)\to\bigsqcup P(U_\alpha)\to\varinjlim P(U_\alpha).
\end{eqnarray*}
\begin{definition}[Germen]
    Dada una sección $s\in P(U)$, se define su \emph{germen} en el punto $x\in S$, denotado por $[s]_x$, como la imagen  $\phi(s)\in P_x$. 
\end{definition}
Notemos que, para un punto $x\in S$, los elementos de $P_x$ son precisamente las secciones en el punto $x$. A $P_x=\{[s]_x\mid s\in PU, x\in U\in\O(x)\}$ se le conoce como el \emph{tallo} de $P$ en $x$. 
\begin{lema}
    La relación ``$\sim$'' es de equivalencia. 
\end{lema}
\begin{proof}
\text{}
\begin{enumerate}
  \item[$\bullet$]  \emph{Reflexividad.} Sea $U\in\O(S)$ y $s\in PU$. Para $x\in U$, tenemos que $U\subseteq U$ y $s=s\mid_U$. Por lo tanto, $s\sim s$.
   \item[$\bullet$] \emph{Simetría.} Para $U,V\in\O(S)$, sean $s\in PU$ y $t\in PV$ tales que $s\sim t$. Entonces, se cumple la definición y dado que existe $W\subseteq U\cap V$, se sigue que también se cumple $t\sim s$.
    \item[$\bullet$]\emph{Transitividad.} Consideremos $U,V,W\in\O(S)$ y, $s\in PU$, $t\in PV$ y $r\in PW$ tales que $s\sim t$ y $t\sim r$. Entonces, existen $Y\subseteq U\cap V$ y $Y'\subseteq V\cap W$ vecindades de $x$ donde $s\mid_Y=t\mid_{Y}$ y $t\mid_{Y'}=r\mid_{Y'}$. Nótese que $Y\cap Y'$ es una vecindad de $x\in U\cap W$. Como $s\mid_Y=t\mid_Y$ implica que $s\mid_{Y\cap Y'}=t\mid_{Y\cap Y'}$. Además, $t\mid_{Y\cap Y'}=r\mid_{Y\cap Y'}$. Por lo tanto, $s\mid_{Y\cap Y'}=r\mid_{Y \cap Y'}$ y $s\sim r$.
    \end{enumerate}
    Ya que $\sim$ es reflexiva, simétrica y transitiva se tiene que es de equivalencia.
\end{proof}

\begin{lema}
    Sea $P^x:\O(x)\to\Con$. Las funciones $\{\germ_x: P(U)\to\O(x)\}$ tales que para cada sección $s\in P(U)$, $\germ_xs=[s]_x$, forman un cocono colímite sobre $\O(x)$. Más aún, este es filtrante.
\end{lema}
\begin{proof}
    Sean $U,V\in\O(x)$. Entonces, se tiene el siguiente diagrama
    \begin{figure}[H]
        \centering
        \includegraphics[width=0.3\linewidth]{img/diagram_2.3.1.png}
    \end{figure}
    Si $s\in PU$ es tal que $s\mid_V=PV$, entonces $[s]_x=[s\mid_V]_x$. Por lo tanto, el diagrama anterior conmuta y $(\germ_x:PU\to P_x)_{U\in \O(x)}$ es un cócono. Ahora, consideremos $(\tau_u: PU\to L)_{U\in\O(x)}$ otro cocono sobre $P^{(x)}$. Veamos que existe una única función $t:P_x\to L$ tal que $t\circ\germ_x=\tau$. Esto es, 
    \begin{figure}[H]
        \centering
        \includegraphics[width=0.45\linewidth]{img/diagram2.3.2.png}
    \end{figure}
    Definamos $t:P_x\to L$ como $t([s]_x)=\tau_U(s)$. Ahora, tomemos $r$ otro representante, i.e., $[s]_x=[r]_x$ implica que existe $W\in U\cap V$ tal que $s\mid_W=r\mid_W$ y $\tau_W(s\mid_W)=\tau_W(r\mid_W)$. Como $\tau$ es un cócono, entonces 
    \begin{eqnarray*}
    \tau_U(s)=\tau_W(s\mid_W)=\tau_W(r\mid_W)=\tau_V(r).
    \end{eqnarray*}
    Por lo tanto, $t$ está bien definida. Ahora, sea $s\in PU$. Entonces, $t\circ\germ_x(s)=t([s]_x)=\tau_U(s)$. Por otro lado, consideremos $l:P_x\to L$ tal que $l\circ\germ_x(s)=\tau$. Si $[s]_x\in P_x$, se tiene que
    \begin{eqnarray*}
        t([s]_x)&=& t\circ\germ_x(s)=\tau(s)\\
        &=& l\circ\germ_x(s)=l([s]_x).
    \end{eqnarray*}
    Por lo tanto, $t$ es única y $\{\germ_x:PU\to P_x\}$ es un cócono límite.
\end{proof}
\begin{lema}
    Cualquier morfismo $h:P\to Q$ de pregavillas (cualquier transformación natural), en cada punto $s\in S$, induce una única función $h_x:P_x\to Q_w$ tal que el diagrama 
    \begin{figure}[H]
        \centering
        \includegraphics[width=0.3\linewidth]{img/diagram2.3.3.png}
    \end{figure}
    conmuta para cualquier $U\in\O(x)$.
\end{lema}
\begin{proof}
    Sea $h:P\to Q$ una transformación natural. Sean $U,V\in\O(x)$ tal que $V\leq U$. Si $s\in PU$, al ser $h$ natural se tiene que $\germ_x\circ h_V(s\mid_V)=\germ_x(h_V(s)\mid_V)$. Por el lema anterior, $\{\germ_x:PU\to PV\}$ es un cócono límite; así, existe una única función $h_x:P_x\to Q_x$ que hace conmutar el diagrama deseado. Además, $h_x([s]_x)=[h_U(s)]_x$. 
\end{proof}
\noindent
Del resultado anterior, tenemos el diagrama conmutativo
    \begin{figure}[H]
        \centering
        \includegraphics[width=0.37\linewidth]{img/diagram2.3.4.png}
    \end{figure}
\noindent
Un resultado inmediato es el siguiente.
\begin{lema}
    La transformación natural dada por $h\to h_x$ es un funtor de $\Con^{\O(x)^{op}}\to \Con$. Dicho funtor es ``tomar germen en $x$''.
\end{lema}

\begin{proof}
Ya hemos probado que la asignación de objetos y flechas está bien definida. Resta verificar que 
preserva identidades y composiciones.
\begin{itemize}
\item Sea $\id_p\colon P\to P$ en $\Con^{\O(x)^{op}}$, Veamos que $(\id_p)_x=\id_{p_x}$. Sea $[s]_x\in P_x$, por el lema anterior,
cualquier morfismo $\id_p\colon P\to P$ induce $\id_{p_x}$. Así,
\[
\id_{p_x}([s]_x)=[\id_p(s)]_x=[s]_x.
\]
\item Sean $h\colon P\to Q$ y $g\colon Q\to R$ morfismos en $\Con^{\O(x)^{op}}$. Veamos que $(g\circ h)_x=g_x\circ h_x$. Sea $[s]_x\in P_x$ y al ser $h$, $g$ 
\item transformaciones naturales, se tiene que
\[
(g\circ h)_x([s]_x)=g_x([h_U(s)]_x)=[g_U(h_U(s))]_x=g_x\circ h_x([s]_x).
\]
\end{itemize}
\end{proof}

Sea 
\begin{eqnarray*}
\Lambda_p=\bigsqcup_{x\in S}P_x=\bigsqcup_{x\in S}\{[s]_x\mid s\in PU\}
\end{eqnarray*} 
la unión disjunta de germenes sobre $x\in S$. Para toda sección $s\in PU$, definamos las siguientes funciones
\begin{eqnarray*}
    p:\Lambda_p&\to& S\\
    \left[s\right]_x&\mapsto& x,\\
    \hat{s}:U&\to&\Lambda_p\\
    x&\mapsto& \left[s\right]_x.
\end{eqnarray*}
\begin{lema}
    El conjunto $B_p:=\{\hat{s}(U)\mid s\in PU, U\in\O(S)\}$ es una base para una topología en $\Lambda_p$.
\end{lema}

\begin{proof}
    Primero veamos que $\bigcup B_p=\Lambda_p$.
    \begin{itemize}
        \item Se cumple que $\bigcup B_p\subseteq \Lambda_p$, para la otra condición consideremos $[s]_x\in \Lambda_p$, entonces $[s]_x=\hat{s}(U)$ para algún $U\in\O(S)$ y $s\in PU$. Por lo tanto, $[s]_x\in B_p$.
        \item Sean $\hat{s}(U)$, $\hat{r}(V)\in B_p$ y supongamos que $\hat{s}(x)=\hat{r}(y)$, entonces $[s]_x=[r]_y$. Por lo tanto $x=y$ y así existe $W\subset U\cap V$ tal que $s\mid_W=r\mid_W$. Luego $\widehat{s\mid_W}(W)$ es un elemento en $B_p$ tal que
        \[
        [s]_x\in\widehat{s\mid_W}(W)\subset \hat{s}(U)\cap \hat{r}(W).        
        \]
    \end{itemize}
\end{proof}

\begin{obs}
Sean $U,V\in \O(S)$ tales que $V\subset U$ y $s\in PU$. Entonces $\hat{s}(V)=\widehat{s\mid_V}(V)$ para todo $s\in PU$.
\end{obs}

\begin{lema}
Sea $\Lambda_p$ dotado con la topología $B_p$. Entonces $p$ y $\hat{s}$ son continuas. Más aún, $p$ es un haz sobre $S$ y $\hat{s}$ es una sección local de $p$.
\end{lema}

\begin{proof}
\begin{itemize}
\item Sea $V\subseteq S$ abierto y sea $[s]_x\in \Lambda_p$ tal que $[s]_x\in p^{-1}(V)$, entonces $x\in U\cap V$. Consideremos $\widehat{s\mid_{U\cap V}}(U\cap V)\in B_p$. Por la observación 
anterior 
\[
[s]_x=[s\mid_{U\cap V}]_x\in \widehat{s\mid_{U\cap V}}(U\cap V)\subseteq p^{-1}(V).
\]
Por lo tanto, $p$ es continua.

\item Para $\hat{s}\colon U\to \Lambda_p$ sea $\hat{r}(V)\in B_p$ un abierto básico y $x\in U$ tal que $\hat{s}(x)\in \hat{r}(V)$. Entonces $[r]_y=[s]_x$ y $x=y$, de aquí que existe $W\subset U\cap V$ tal que 
$s\mid_W=r\mid_W$. Por lo tanto, $\hat{s}(W)$ es un abierto básico tal que 
\[
[s]_x\in \hat{s}(W)\subseteq \hat{r}(V).
\]
De esta manera se concluye que $\hat{s}$ es continua.
\end{itemize}
Para ver que $\hat{s}$ es una sección local de $p$, basta notar que para todo $x\in U$ se tiene que $p\circ \hat{s}(x)=p([s]_x)=x$, por lo tanto 
$p\circ \hat{s}=\id_U$, es decir, el diagrama
\[\begin{tikzcd}
	U && {\Lambda_p} \\
	& S
	\arrow["{\hat{s}}", from=1-1, to=1-3]
	\arrow["i"', hook, from=1-1, to=2-2]
	\arrow["p", from=1-3, to=2-2]
\end{tikzcd}\]
conmuta.
\end{proof}

\begin{definition}
Sea $f\in \Top$, decimos que $f$ es un \emph{homeomorfismo} si tiene una inversa continua.
\end{definition}

\begin{lema}
    $\hat{s}\colon U\to \hat{s}(U)$ es un homeomorfismo.
\end{lema}

\begin{proof}
Consideremos $V\subset U$ abierto de $S$, entonces $\hat{s}(V)\in B_p$ y al ser continua, $\hat{s}$ es abierta.

Supongamos que $\hat{s}(x)=\hat{s}(y)$, entonces $[s]_x=[s]_y$, es decir, $x=y$. Por lo tanto $\hat{s}$ es inyectiva. Ahora, sea $W\subset \hat{s}(U)$ abierto, entonces existe $V\subset U$ abierto tal que $\hat{s}(V)=W$. Por lo tanto, $\hat{s}^{-1}(W)=V$ y así $\hat{s}$ es biyectiva. 
Finalmente, como $\hat{s}$ es continua y abierta, se sigue que es un homeomorfismo.
\end{proof}

\begin{lema}
Si $h\colon P\to Q$ es un morfismo de pregavillas, la unión ajena de funciones $h_x\colon P_x\to Q_x$ es un morfismo de hazes continuos $h_p\colon \Lambda_P\to \Lambda_Q$.
\end{lema}

\begin{proof}
Primero verificamos que $\langle h_x: P_x\to Q_x\rangle_{x\in S}$ es una función continua.

Sean $\hat{s}(U)\in B_Q$ un abierto básico de $\Lambda_Q$ y $[t]_x\in \Lambda_p$ tal que $h_x([t]_x)\in \hat{s}(U)$. Entonces, $[h_U(t)]_x=[s]_x$ y por lo tanto existe uan vecindad $V$ de $x$ tal que $V\subset U\cap W$ 
de forma que $h_U(t)\mid_V=s\mid_V$ y $\widehat{h_U(t)\mid_V}(V)=\widehat{h_U(t)}(V)$. Por lo tanto, $\widehat{h_U(t)}(V)\in B_Q$.

También tenemos que $[t]_x\in \hat{t}(V)=\widehat{t\mid_V}(V)$. Como $h$ es transformación natural, se tiene que el diagrama 
\[\begin{tikzcd}
	PU & QU \\
	PV & QV
	\arrow["{h_U}", from=1-1, to=1-2]
	\arrow[from=1-1, to=2-1]
	\arrow[from=1-2, to=2-2]
	\arrow["{h_V}"', from=2-1, to=2-2]
\end{tikzcd}\]
conmuta. Así, $h_V(t\mid_V)=h_U(t)\mid_V=s\mid_V$ para todo $t\in PU$. Por lo tanto, $\langle h_x: P_x\to Q_x\rangle_{x\in S}$ es continua.

Sea $[s]_x\in \Lambda_P$, entonces $p([s]_x)=x=q([h_U(s)]_x)$ y $\langle h_x: P_x\to Q_x\rangle_{x\in S}$ es un morfismo de hazes. 
\end{proof}

\begin{lema}
$P\to \Lambda_P$ es un funtor de pregavillas a hazes.
\end{lema}

\begin{proof}
Sea $\id_p\colon P\to P$ la identidad en $\Con^{\O(S)^{op}}$, entonces $\id_{p_x}$ es la función identidad para cada $x\in S$. De aquí que $\langle \id_{p_x}: P_x\to P_x\rangle_{x\in S}$ es la identidad en $\Lambda_P$. Por lo tanto, $P\to \Lambda_P$ preserva identidades.

Sean $h\colon P\to Q$ y $g\colon Q\to R$ morfismos de pregavillas, entonces $g_x\circ h_x=(g\circ h)_x$ para cada $x\in S$. Por lo tanto, $\langle g_x\circ h_x: P_x\to R_x\rangle_{x\in S}$ es la composición de morfismos de hazes. Así, $P\to \Lambda_P$ preserva composiciones.

Por lo tanto, $P\to \Lambda_P$ es un funtor de pregavillas a hazes.
\end{proof}

\begin{definition}
Una función $f\colon S\to T$ entre espacios topológicos es un homomorfismo local si para cada $x\in S$ existe un abierto $U\subset S$ tal que $x\in U$ y $f\mid_U$ es un homeomorfismo.
\end{definition}

\begin{lema}
Toda sección de un homeomorfismo local es una función abierta.
\end{lema}

\begin{proof}
Sea $p\colon T\to S$ un homeomorfismo local, $s\colon U\to T$ una sección de $p$ y $V\subset U$ un abierto. Sea $y\in s(V)$ y como $p$ es un homeomorfismo local, existe un abierto $W\subset S$ tal que $p\mid_W$ es un homeomorfismo. Entonces, $p(W)$ es un abierto de $T$ y así $p(W)\cap V$ es abierto. Al ser $s$ una sección de $p$, se tiene que $p(W\cap s(V))=p(W)\cap V$. Además, $p\mid_W$ es una función abierta, $W\cap s(V)$ es una vecindad abierta de $x$ contenida en $s(V)$. Por lo tanto, $s(V)$ es abierto en $T$.
\end{proof}

\begin{lema}
La función $p\colon \Lambda_P\to S$ es un homeomorfismo local.
\end{lema}

\begin{proof}
Sea $[s]_x\in \Lambda_P$. El abierto básico $\hat{s}(U)$ contiene a $[s]_x$ y al ser $\hat{s}$ una sección de $p$, $p(\hat{s}(U))=U$. Luego, $\hat{s}\colon U\to \hat{s}(U)$ es un homeomorfismo. Por lo tanto, $p\mid_{\hat{s}(U)}\colon \hat{s}(U)\to p\hat{s}(U)$ es un homeomorfismo.
\end{proof}

\section{$\H L/S\cong \Gav(S)$}

\section{Cambio de base}

Recordemos que para $S\in \Top$, $\mathcal{O}S$ es la categoría cuyos objetos son los abiertos de $S$ y los morfismos son las 
inclusiones. De esta manera, para $U, V\in \mathcal{O}S$ con $V\subseteq U$, $\mathcal{O}S^{\op}$ tiene las flechas $U\to V$.\\

Así, para $S, T\in \Top$ tomamos $\Gav(S), \Gav(T)$ parta dar la siguiente definición.

\begin{definition}\label{Def 2.5.1}
Consideremos $f_*\colon \Gav(S)\to \Gav(T)$ dado por 
\[
f_*(F(V))=F(f^{-1}(V))
\]
para $V\in \mathcal{O}T$ y $F\in \Gav(S)$. Además, si $h\colon F\to G$ es una transformación natural en $\Gav(S)$ y $V\in \mathcal{O}S$,
entonces 
\[
f_*(h_V)=h_{f^{-1}(V)}.
\]
\end{definition}

\begin{lema}\label{Lem 2.5.2}
$f_*F$ es una gavilla.
\end{lema}

\begin{proof}
Sea $V\in \mathcal{O}T$ y $\{V_i\}_{i\in I}$. Debemos verificar que
\begin{equation}\label{Eq 2.5.1}
\begin{tikzcd}
	{f_*F(V)} & {\prod_{i\in I}f_*F(V_i)} & {\prod_{i,j\in I}f_*F(V_i\cap V_j)}
	\arrow["\epsilon", from=1-1, to=1-2]
	\arrow["p", shift left=2, from=1-2, to=1-3]
	\arrow["q"', shift right=2, from=1-2, to=1-3]
\end{tikzcd}
\end{equation}
es un igualador.\\

Al ser $f\colon S\to T$ una función continua, $f^{-1}(V)\in \mathcal{O}S$.
Consideremos $U=f^{-1}(V)$ y $\{U_i\}_{i\in I}$ con $U_i=f^{-1}(V_i)$. 
Luego, $U=\bigcup_{i\in I}U_i$ y por \ref{Eq 2.5.1} tenemos
\[\begin{tikzcd}
	{F(U)} & {\prod_{i\in I}F(U_i)} & {\prod_{i,j\in I}F(U_i\cap U_j)}
	\arrow["\epsilon", from=1-1, to=1-2]
	\arrow["p", shift left=2, from=1-2, to=1-3]
	\arrow["q"', shift right=2, from=1-2, to=1-3]
\end{tikzcd}\]
es un igualador pues, por hipótesis, $F$ es una gavilla.
\end{proof}

\begin{lema}\label{Lem 2.5.3}
$f_*\colon \Gav(S)\to \Gav(T)$ es un funtor.
\end{lema}

\begin{proof}
Consideremos $F\in \Gav(S)$ y sea $\id_F\colon F\to F$ la transformación natural identidad. Para $V\in \mathcal{O}T$
tenemos que 
\[
f_*\id_F(V)\colon f_*F(V)\to f_*F(V)
\]
es la identidad para todo $V\in \mathcal{O}T$. Por la Definición \ref{Def 2.5.1}
\[
\id_F(f^{-1}(V))\colon F(f^{-1}(V))\to F(f^{-1}(V))
\]
y por lo tanto, $f_*$ respeta identidades.\\

Ahora, sean $F, G, H\in \Gav(S)$ y $h\colon F\to G$, $g\colon G\to H$ transformaciones naturales. Entonces, para $V\in \mathcal{O}T$
\[\begin{tikzcd}
	{f_*F(V)} & {f_*G(V)} & {f_*H(V)}
	\arrow["{f_*k(V)}", from=1-1, to=1-2]
	\arrow["{f_*l(V)}", from=1-2, to=1-3]
\end{tikzcd}\]
es la composición $f_*(L\circ h)$. Por definición, 
\[\begin{tikzcd}
	{F(f^{-1}(V))} && {G(f^{-1}(V))} && {H(f^{-1}(V))}
	\arrow["{k_{f^{-1}(V)}}", from=1-1, to=1-3]
	\arrow["{l_{f^{-1}(V)}}", from=1-3, to=1-5]
\end{tikzcd}\]
Como $k, l$ son transformaciones naturales, 
\[
(l\circ k)_{f^{-1}(v)}\colon F(f^{-1}(V))\to H(f^{-1}(V))
\]
y por definición 
\[
f_*(L\circ h)(V)= (l\circ k)_{f^{-1}(V)}.
\]
Por lo tanto, $f_*$ respeta composiciones.
\end{proof}

\begin{definition}\label{Def 2.5.4}
Consideremos $f^*\colon \Top/T\to \Top/S$ dada por la asignación 
\begin{enumerate}
    \item para $(p\colon E\to T)\in \Top/T$, $f^*p\colon f^*E\to S$ (o simplemente $p^*$) es el pullback del diagrama
    \[\begin{tikzcd}
	{f^*E} & E \\
	S & T
	\arrow["e", from=1-1, to=1-2]
	\arrow["{p^*}"', from=1-1, to=2-1]
	\arrow["p", from=1-2, to=2-2]
	\arrow["f"', from=2-1, to=2-2]
    \end{tikzcd}\]
    \item para $k\colon p\to p'$ un morfismo en $\Top/T$, $f^*k\colon p^*\to p'^*$ (o simplemente $k^*$) es el morfismo 
    inducido por la propiedad universal del pullback que hace conmutar el diagrama
    \[\begin{tikzcd}
	{f^*E} & E \\
	& {f^*E'} & {E'} \\
	& S & T
	\arrow["e", from=1-1, to=1-2]
	\arrow["{k^*}", dashed, from=1-1, to=2-2]
	\arrow["{p^*}"', from=1-1, to=3-2]
	\arrow["k", from=1-2, to=2-3]
	\arrow["{e'}"', from=2-2, to=2-3]
	\arrow["{p'^*}", from=2-2, to=3-2]
	\arrow["{p'}", from=2-3, to=3-3]
	\arrow["f"', from=3-2, to=3-3]
    \end{tikzcd}\]
\end{enumerate}
\end{definition}

\begin{lema}\label{Lem 2.5.5}
    Si $p\colon E\to T$ es un homeomorfismo local, entonces $p^*\colon f^*E\to S$ es un homeomorfismo local.
\end{lema}

\begin{proof}
Sabemos que $f\colon S\to T$ es continua y $p\colon E\to T$ un homeomorfismo local. Además, 
\[
f^*E=\{(s,e)\in S\times E\mid f(s)=p(e)\}
\]
con la topología de subespacio de $S\times E$. Consoideremos $\langle x, e\rangle\in f^*E$, al ser $p$ un homeomorfismo local, para todo 
$e\in E$ existe un abierto $U\subset E$ tal que $e\in U$ y $p\mid_U\colon U\to p(U)$ es un homeomorfismo, de aquí que $p(U)\in \mathcal{O}T$ y $f^{-1}(p(U))\in \mathcal{O}S$. Luego,
\[
f(s)=p(e)\in p(U)\Rightarrow s\in f^{-1}(p(U)).
\]
Para que $p^*=f^*p$ sea un homeomorfismo local necesitamos que para todo $x\in f^*E$ exista un abierto $W\subset f^*E$ tal que $x\in W$ y $p^*\mid_W\colon W\to p^*(W)$ es un homeomorfismo. 
Consideremos $f^{-1}(p(U))\times U$ un abierto que contiene a $\langle x, e\rangle$ en $S\times E$. Asi, para 
\[
W=f^{-1}(p(U))\times U\cap f^*E,
\]
si $\langle s, e\rangle\in W$, entonces $f(s)=p(e)$. De aquí que debemos probar que $p^*\colon W\to f^{-1}(p(U))$ es homeomorfismo.\\

Primero veamos que es suprayectiva. Sea $z\in f^{-1}(p(U))$, entonces $f(z)\in p(U)$ y como $p\mid_U\colon U\to p(U)$ es suprayectiva, existe $e\in U$ tal que $p(e)=f(z)$. Por lo tanto,
\[
\langle z, e\rangle\in W\quad\text{y}\quad p^*(\langle z, e\rangle)=z.
\]

Para la inyectividad, si $p^*(\langle z, v\rangle)=p^*(\langle z', v' \rangle)$, entonces $z=z'$ y como $f(z)=p(v)$ y $f(z')=p(v')$, se tiene que $p(v)=p(v')$. Al ser $p\mid_U\colon U\to p(U)$ inyectiva, se sigue que $v=v'$. 
Por lo tanto, $\langle z, v\rangle=\langle z', v'\rangle$ y así $p^*$ es inyectiva.\\

Notemos que $p^*$ es continua por construcción. Así, para ver que es abierta, sean $\langle s,e\rangle\in f^*E$ y $U\subseteq E$ abierto tal que $p\mid_U\colon U\to p(U)$ es un homeomorfismo y $p(U)\in \mathcal{O}T$. $W$ es 
abierto en $f^*E$ y contiene a $\langle s,e\rangle$. Al ser $p\mid_U$ un homeomorfismo, existe $(p\mid_U)^{-1}\colon p(U)\to U$ y para $z\in f^{-1}(p(U))$ se tiene que
\[
h\colon f^{-1}(p(U))\to W\quad \mbox{dada por }\quad h(z)=\langle z, (p\mid_U)^{-1}(f(z))\rangle    
\]
es la inversa de $p^*\mid_W$ y es continua (pues $f$ y $p\mid_U$ son continuas). Por lo tanto, $p^*\mid_W$ es un homeomorfismo y así $p^*$ es un homeomorfismo local.
\end{proof}

\begin{lema}\label{Lem 2.5.6}
$f^*\colon \Top/T\to \Top/S$ es un funtor.
\end{lema}

\begin{proof}
Sea $p\colon E\to T$ un objeto en $\Top/T$ y $\id_p\colon p\to p$ la identidad en $\Top/T$. Entonces, por la Definición \ref{Def 2.5.4}, $f^*\id_p\colon p^*\to p^*$ es la identidad en $\Top/S$. Por lo tanto, $f^*$ preserva identidades.\\

Sean $k\colon p\to p'$ y $k'\colon p'\to p''$ morfismos en $\Top/T$. EL haz composición es $k'\circ k$ y por la Definición \ref{Def 2.5.4}, $(k'\circ k)^*\colon p^*\to p''^*$ es el morfismo inducido por la propiedad universal del pullback que hace conmutar el diagrama
\[\begin{tikzcd}
f^*E \arrow[rrr, "e"] \arrow[rd, "k^*", dashed] \arrow[rdd, "k'^* \circ k^*" description] \arrow[rddd, "p^*"', bend right] &                                                  &                                                           & E \arrow[ld, "k"'] \arrow[lddd, "p", bend left] \\
                                                                                                                           & f^*E' \arrow[d, "k'^*"', dashed] \arrow[r, "e'"] & E' \arrow[d, "k'"]                                        &                                                 \\
                                                                                                                           & f^*E'' \arrow[d, "p''^*"'] \arrow[r, "e''"]      & E'' \arrow[d, "p''"] \arrow[ruu, "k'\circ k" description] &                                                 \\
                                                                                                                           & S \arrow[r, "f"]                                 & T                                                         &                                                
\end{tikzcd}\]
Por lo tanto, respecta composiciones.
\end{proof}

De esta manera, para $(f\colon S\to T)\in \Top$, tenemos un funtor de $\Gav(t)\to \Gav(S)$ dado por

\begin{equation}\label{Diagrama composicion}
    \begin{tikzcd}
	{\Gav(T)} & {\Top /T} & {\Top /S} & {\Gav(S)}
	\arrow["\Lambda", from=1-1, to=1-2]
	\arrow["{f^*}", from=1-2, to=1-3]
	\arrow["\Gamma", from=1-3, to=1-4]
\end{tikzcd}
\end{equation}

Para $G\in \Gav(T)$, notemos que $\Lambda G=\{[s\in GV]_y\mid y\in T, s\in GV\}$ y el homeomorfismo local $p\colon \Lambda G\to T$ es tal que
$p([s\in GV]_y)=y$. Así
\[
f^*\Lambda G=\{(x, [s\in GV]_y)\in S\times \Lambda G\mid f(x)=p([s\in GV]_y)=y\}
\]
y el homeomorfismo local $p^*\colon f^*\Lambda G\to S$ es tal que $p^*(x, [s\in GV]_y)=x$.\\

Si $p([s\in GV]_y)=y$, entonces $f^*\Lambda G=\{(x,[s\in GV]_y)\in S\times \Lambda G\mid f(x)=y\}$ y 
\[
f^*\Lambda G=\{(x, [s\in GV]_{f(x)})\in S\times \Lambda G\mid x\in S, s\in GV\}.
\]

Por último, $\Gamma f^* \Lambda G=\Gamma_{p^*}$. Así, 
\[
\Gamma_{p^*}(U)=\{t\colon U\to f^*\Lambda G \mbox{ continia}\mid p^*\circ t=i_U\}.
\]
con $U\in \mathcal{O}S$.\\

Cada sección $t\in \Gamma_{p^*}(U)$ se describe a través de $t'\colon U\to \Lambda G$ continua, como 
\[\begin{tikzcd}
	U \\
	& {f^*\Lambda G} & {\Lambda G} \\
	& S & T
	\arrow["t"', dashed, from=1-1, to=2-2]
	\arrow["{t'}", from=1-1, to=2-3]
	\arrow["i"', from=1-1, to=3-2]
	\arrow["e"', from=2-2, to=2-3]
	\arrow["{p^*}", from=2-2, to=3-2]
	\arrow["p", from=2-3, to=3-3]
	\arrow["f"', from=3-2, to=3-3]
\end{tikzcd}\]
La sección $\hat{s}\colon V\to \Lambda G$ induce una sección $t_s$ de $p^*$ como 
\[\begin{tikzcd}
	{f^{-1}(V)} & V \\
	& {f^*\Lambda G} & {\Lambda G} \\
	& S & T
	\arrow["f", from=1-1, to=1-2]
	\arrow["{t_s}", dashed, from=1-1, to=2-2]
	\arrow["i"', from=1-1, to=3-2]
	\arrow["{\hat{s}}", from=1-2, to=2-3]
	\arrow["e"', from=2-2, to=2-3]
	\arrow["{p^*}", from=2-2, to=3-2]
	\arrow["p", from=2-3, to=3-3]
	\arrow["f"', from=3-2, to=3-3]
\end{tikzcd}\]
para $V\in \mathcal{O}T$ y $s\in GV$.\\

Si $x\in f^{-1}(V)$, entonces $p(\hat{s}(f(x)))=p([s\in GV]_{f(x)})$ y $t_s(x)=(x, [s\in GV]_{f(x)})$.

\begin{definition}\label{Def 2.5.7}
Denotamos por $f^*\colon \Gav(T)\to \Gav(S)$ al funtor compuesto $f^*=\Gamma\circ f^*\circ \Lambda$.
\end{definition}

\begin{lema}\label{Lema 2.5.8}
Consideremos $X\in \Top$. Una función $k\colon f^*\Gamma G\to X$ es continua si y solo si 
\[\begin{tikzcd}
	{f^{-1}(W)} & {f^*\Lambda G} & X
	\arrow["{t_r}", from=1-1, to=1-2]
	\arrow["k", from=1-2, to=1-3]
\end{tikzcd}\]
es continua para todo $W\subseteq T$ y para toda $r\in GW$.
\end{lema}

\begin{proof}

\end{proof}

\section{Morfismos ultrafinitos}

Sabemos que si ($f\colon S\to T)\in \Top$, $f$ induce la siguiente adjunción
\[
\begin{tikzcd}
\Gav(S) \arrow[rr, "f_*", bend left] & \perp & \Gav(T) \arrow[ll, "f^*", bend left]
\end{tikzcd}
\]
donde $f^*\colon \Gav(T)\to \Gav(S)$ es exacto izquierdo.\\

Además, para el funtor $f_*\colon \Gav(S)\to \Gav(T)$ tenemos que si $V\in \mathcal{O}T$ y $F\in \Gav(S)$, $f_*F(V)=F(f^{-1}(V))$ 
y para $(h\colon F\to G)\in \Gav(S)$, $f_h(V)=h_{f^{-1}(V)}$. Queremos ver que $f_*$ preserva:
\begin{enumerate}
\item Objeto inicial.
\item Epimorfismos.
\item Sumas finitas.
\end{enumerate}

Primero, si $\mathbf{0}\in \Gav(S)$ queremos ver que para todo $\emptyset \neq U\in \mathcal{O}S$, $\mathbf{0}(U)=\emptyset$. Sea $\mathbf{0}\colon \mathcal{O}S^{\op}\to \Con$
la pregavilla tal que $\mathbf{0}(U)=\emptyset$ para todo $\emptyset \neq U\in \mathcal{O}S$. Notemos que 
\[
\Lambda (\mathbf{0})=\{[s\in \mathbf{0}U]_x\mid x\in S, s\in \mathbf{0}U\}
\]
y $\mathbf{0}U=\emptyset$, entonces $\Lambda(\mathbf{0})=\emptyset$. Luego, $\Gamma_\emptyset U=\{s\colon U\to \emptyset\mid s \mbox{ es sección de }i\colon \emptyset \to S\}$. Por lo tanto, 
$\Gamma_\emptyset U=\emptyset$ cuando $U\neq \emptyset$.

\begin{lema}\label{Prop 3.0.11}
    Sea $(f\colon S\to T)\in \Top$, entonces el funtor $f_*\colon \Gav(S)\to \Gav(T)$ preserva el objeto inicial si y solo si $s(S)$ es denso en $T$.
\end{lema}

\begin{proof}
\begin{description}
\item[$\Rightarrow)$] Supongamos que $f_*$ preserva el objeto inicial. Sea $V\in \mathcal{O}T$ tal que $V\neq\emptyset$. Entonces
\[
f_*\mathbf{0}(V)=\mathbf{0}(f^{-1}(V))=\emptyset\Rightarrow f^{-1}(V)\neq \emptyset.
\]
Notemos que $f(S)$ es denso en $T$ si para todo $V\in \mathcal{O}T$ se tiene que $V\cap f(S)\neq \emptyset$. Si $f^{-1}(V)\neq \emptyset$, existe $x\in S$ tal que $f(x)\in V$. Por lo tanto, $V\cap f(S)\neq \emptyset$ y así $f(S)$ es denso en $T$.

\item[$\Leftarrow)$] Supongamos que $f(S)$ es denso en $T$. Sea $V\in \mathcal{O}T$ tal que $V\neq \emptyset$, entonces $V\cap f(S)\neq \emptyset$. Por lo tanto, existe $x\in S$ tal que $f(x)\in V$ y así $f^{-1}(V)\neq \emptyset$. Luego, 
$\mathbf{0}(f^{-1}(V))=\emptyset$ y por lo tanto $f_*\mathbf{0}(V)=\emptyset$. Así, $f_*$ preserva el objeto inicial.
\end{description}
\end{proof}

Ahora, enunciaremos las condiciones necesarias y suficientes para que $f_*$ preserve epimorfismos.

\begin{lema}\label{Lema 3.0.12}
Si para cada $V\in \mathcal{O}T$, $y\in V$ y $\{U_i\}_{i\in I}\subseteq \mathcal{O}S$ una cubierta abierta de $f^{-1}(V)$, existen $W\in\mathcal{O}T$ tal que $y\in W$ y $\{W_i\}_{i\in I}\subseteq \mathcal{O}S$ una cubierta 
abierta ajena de $f^{-1}(W)$ con la propiedad que $W_i\subseteq U_i$ para todo $i\in I$. Enotnces para todo $q\colon E\to f^{-1}(V)$ epimorfismo en $HL/S$, existe $t\colon f^{-1}(W)\to E$ tal que el diagrama 
\[\begin{tikzcd}
	{f^{-1}(W)} && {f^{-1}(V)} \\
	& E
	\arrow[hook, from=1-1, to=1-3]
	\arrow["t"', from=1-1, to=2-2]
	\arrow["q"', from=2-2, to=1-3]
\end{tikzcd}\]
conmuta.
\end{lema}

\begin{proof}
Sea $q\colon E\to f^{-1}(V)$ un epimorfismo en $HL/S$. Si $x\in f^{-1}(V)$, entonces tenemos abiertos $U_x\subset f^{-1}(V)$ y $U_x'\subset E$ tales que 
\[
q\colon U_x'\to U_x
\]
es un homeomorfismo. Así, $\{U_x\}_{x\in f^{-1}(V)}$ es una cubierta abierta de $f^{-1}(V)$. Por hipótesis, existen $W\in \mathcal{O}T$ tal que $y\in W$ y $\{W_x\}_{x\in f^{-1}(V)}$ una cubierta abierta ajena de $f^{-1}(W)$ con la propiedad que $W_x\subseteq U_x$ para todo $x\in f^{-1}(V)$. Además 
$q_{\mid_{U_x'}}\colon U_x'\to q(U_x')$ es un homeomorfismo, entonces definimos 
\[t_x=(q_{\mid_{U_x}})^{-1}\colon W_x\to E.
\]
y así tomamos $t\colon f^{-1}(W)\to E$ dada por $t_{\mid_{W_x}}=t_x$. Por construcción, $q\circ t=i_{f^{-1}(W)}$ y el diagrama conmuta.
\end{proof}

Para la prueba del siguiente resultado, hacemos uso de los siguientes criterios. Estos pueden ser encontrados en \textbf{Citar los criterios}.

\begin{description}
\item[Criterio 1:] Sea $h\colon F\to G$ un morfismo en $\Gav(S)$. Entonces $h$ es un epimorfismo si y solo si para todo $U\in \mathcal{O}S$ y $s\in G(U)$, existe una cubierta abierta 
$\{U_i\}_{i\in I}$ de $U$ y para cada $i\in I$ una sección $t_i\in F(U_i)$ tal que $h_{U_i}(t_i)=s_{\mid_{U_i}}$, en otras palabras, para $y\in U$
\[
[s\in G(U)]_y=[h_{U_i}(t_i)\in G(U_i)]_y.
\]

\item[Criterio 2:] $\Lambda f_*\Gamma(h)$ es un epimorfismo en $HL/T$ si y solo si para todo $x\in S$ y germen $s_x$ de $G$ en $x$ existe un germen $t_x$ en $F$ con la propiedad que 
\[[h_{U}(t_x)\in G(U)]_{f(x)}=[s_x\in G(U)]_{f(x)}\]
para algún abierto $U\in \mathcal{O}T$ que contiene a $f(x)$.
\end{description}

\begin{lema}\label{Lema 3.0.13}
Sea $h\colon E\to E'$ tal que el diagrama
\[\begin{tikzcd}
	E && {E'} \\
	& S
	\arrow["h", from=1-1, to=1-3]
	\arrow["p"', from=1-1, to=2-2]
	\arrow["{p'}"', from=2-2, to=1-3]
\end{tikzcd}\]
conmuta. Entonces $\Lambda f_*\Gamma(h)$ es un epimorfismo si y solo si para todo $V\in \mathcal{O}T$, $y\in V$ y cualquier sección $s\colon f^{-1}(V)\to E'$ de $p'$, existen $W\in \mathcal{O}T$ con $y\in W$ y una sección $t\colon f^{-1}(W)\to E$ de $p$ tal que el diagrama
\begin{equation}\label{Diagrama Lema 3.0.13}
\begin{tikzcd}
    {f^{-1}(W)} & {f^{-1}(V)} \\
    E & {E'}
    \arrow["t", from=1-1, to=2-1]
    \arrow["s"', from=1-2, to=2-2]
    \arrow["h", from=2-1, to=2-2]
    \arrow[hook, from=1-1, to=1-2]
\end{tikzcd}
\end{equation}
conmuta.
\end{lema}

\begin{proof}
\begin{description}
\item[$\Rightarrow)$] Para $h\colon E\to E'$ tenemos que $\Lambda f_*\Gamma(h)\colon \Lambda f_*\Gamma E\to \Lambda f_*\Gamma E'$ es un morfismo en $HL/T$ tal que si $r\in \Gamma_p(f^{-1}(V))$, entonces 
\[
\Lambda f_*\Gamma(h)([r\in \Gamma_{p}(f^{-1}(V))]_y)=[h\circ r\in \Gamma_{p'}(f^{-1}(V))]_y. 
\]
Sean $V\in \mathcal{O}T$, $y\in V$ y $s\colon f^{-1}(V)\to E'$ una sección de $p'$. 
Supongamos que $\Lambda f_*\Gamma(h)$ es un epimorfismo, entonces existe $[t\in \Gamma_{p}(f^{-1}(V))]_y\in \Lambda f_*\Gamma(E)$ tal que
\begin{equation}\label{Eq Lema 3.0.13}
[h\circ t\in \Gamma_{p'}f^{-1}(V)]_y=[s\in \Gamma_{p'}f^{-1}(V)]_y.
\end{equation}
De aquí, por el Criterio 1, existe un abierto $W\subset V$ con $y\in W$ tal que $h\circ t_{\mid_{f^{-1}(W)}}=s_{\mid_{f^{-1}(W)}}$. Por lo tanto, el diagrama (\ref{Diagrama Lema 3.0.13}) conmuta.

\item[$\Leftarrow)$] Supongamos que para todo $V\in \mathcal{O}T$, $y\in V$ y cualquier sección $s\colon f^{-1}(V)\to E'$ de $p'$, existen $W\in \mathcal{O}T$ con $y\in W$ y una sección $t\colon f^{-1}(W)\to E$ de $p$ tal que el diagrama (\ref{Diagrama Lema 3.0.13}) conmuta. 
De aquí que \ref{Eq Lema 3.0.13} se cumple y 
\[
[s_{\mid_{f^{-1}(V)}}\in \Gamma_{p'}(f^{-1}(W))]_y=[s\in \Gamma_{p'}(f^{-1}(V))]_y.
\]
Por lo tanto, por el Criterio 2, $\Lambda f_*\Gamma(h)$ es un epimorfismo.
\end{description}
\end{proof}

\begin{prop}\label{Lema 3.0.14}
Consideremos $(f\colon S\to T)\in \Top$. El funtor $f_*\colon \Gav(S)\to \Gav(T)$ preserva epimorfismos si y solo si para todo $V\in \mathcal{O}T$, $y\in V$ y $\{U_i\}_{i\in I}\subseteq \mathcal{O}S$ una cubierta abierta de $f^{-1}(V)$, existe $W\in\mathcal{O}T$ tal que $y\in W$ y $\{W_i\}_{i\in I}\subseteq \mathcal{O}S$ 
una cubierta abierta ajena de $f^{-1}(W)$ con la propiedad que $W_i\subseteq U_i$ para todo $i\in I$.
\end{prop}

\begin{proof}

\end{proof}
