\chapter{Gavillas sobre Marcos}

Lo que se ha abordado hasta ahora es la teoría de haces y gavillas de conjuntos sobre espacios topológicos. En \cite{Merino2002}, se encargan de trasladar las nociones de morfismo étale y morfismos ultrafinitos a un contexto más general, el de los locales. En este capítulo se presentan algunas de las ideas principales de dicha referencia 
pero viéndolo desde el enfoque de marcos. Para tener más información sobre teoría de marcos, se puede consultar (\textbf{citar las notas de Ángel y bibliografía básica sobre teoría de marcos}).
\section{Marcos y gavillas}

En el sentido categórico, trabajar con locales y trabajar con marcos, en esencia podría parecer lo mismo, ya que la categoría de locales es la categoría opuesta a la categoría de marcos. Sin embargo, en la práctica, trabajar con marcos suele ser más sencillo. 
Lo primero que haremos es dar las herramientas necesarias para trasladar las nociones vistas en capítulos anteriores al lenguaje de marcos.

\begin{definition}
    Un \emph{marco} es una retícula completa $(A, \leq, \bigvee, \wedge, 0, 1)$ que satisface la siguiente propiedad distributiva:
    \[
    a\wedge \bigvee x=\bigvee \{a\wedge b\mid x\in X\}
    \]
    para todo $a\in A$ y $X\subseteq A$.
\end{definition}

\begin{definition}
    Sean $A$ y $B$ dos marcos. Un \emph{morfismo de marcos} es una función monótona $f\colon A\to B$ que preserva
    la estructura de la retícula, es decir, para todo $X\subseteq A$ y $a,b\in A$ se cumple:
    \begin{itemize}
        \item $f(\bigvee X)=\bigvee \{f(x)\mid x\in X\}$.
        \item $f(a\wedge b)=f(a)\wedge f(b)$.
        \item $f(0)=0$ y $f(1)=1$.
    \end{itemize}
\end{definition}

De esta manera, la categoría de marcos $\Frm$ es la categoría cuyos objetos son marcos y cuyos morfismos son morfismos de marcos, y como mencionamos antes, la categoría de locales $\Loc$ es la categoría opuesta a $\Frm$.\\

Si $f:A\to B$ es un morfismo de marcos, entonces $f$ tiene un adjunto derecho $f_*:B\to A$ y cumple la siguiente propiedad:
\[
f(a)\leq b \iff a\leq f_*(b)
\]
para todo $a\in A$ y $b\in B$. De manera general, podemos calcular el adjunto derecho de un morfismo de marcos $f:A\to B$ como:
\[
f_*(b)=\bigvee \{a\in A\mid f(a)\leq b\}
\]

\begin{definition}
Sea $A\in \Frm$. La operación \emph{implicación} (o también conocida como implicación de Heyting), denotada por $(a\succ b)$ para $a, b\in A$, se define por:
\[
a\wedge c \leq b \iff c \leq (a\succ b).
\]
\end{definition}

Consideremos los morfismos $f\colon A\to A$ y $f_*\colon A\to A$, dados por
\[
f(x)=a\wedge x \quad \text{y} \quad f_*(x)=(a\succ x),
\]
entonces $f(x)\leq y$ si y solo si $x\leq f_*(y)$, es decir, $x\wedge a\leq y$ si y solo si $x\leq (a\succ y)$. En otras palabras,
implicar es adjunto derecho de calcular ínfimo. Por lo tanto, la implicación se puede calcular como
\[
(a\succ b)=\bigvee \{x\in A\mid a\wedge x \leq b\}.
\]

El siguiente resultado proporciona algunas propiedades básicas de la implicación.

\begin{prop}
    Consideremos $A$ un marco y $a,b,c\in A$. Entonces 
    \begin{itemize}
        \item $1\succ a=a$.
        \item $a\succ (b\wedge c)=(a\succ b)\wedge (a\succ c)$.
        \item $(a\vee b)\succ c=(a\succ c)\wedge (b\succ c)$.
        \item $a\leq b$ si y solo si $1\leq (a\succ b)$.
        \item $a\leq (b\succ c)$ si y solo si $a\wedge b \leq c$.
    \end{itemize}
\end{prop}

\subsection{Núcleos y cocientes en marcos}
Cuando trabajamos en $\Top$, la noción de subespacio puede ser capturada mediante la noción de sublocal. A su vez, los sublocales pueden ser estudiados de manera 
algebraica mediante lo que en marcos se conoce como \emph{núcleos}. Esta subsección explicara brevemente la relación entre núcleos y sublocales.

\subsection{Morfismos abiertos}

\begin{definition}[Morfismo abierto]
    Sea $f:A\to B$ un morfismo de marcos. Decimos que es abierto si existe $f_l:B\to A$, un morfismo de copos tal que:
    \begin{itemize}
        \item $f_l\dashv f$.
        \item Para cada $a\in A$ y $b\in B$ se tiene la identidad de Frobenius 
            $$f_l(b\wedge f(a))=f_l(b)\wedge a.$$
    \end{itemize}
\end{definition}

\begin{obs}
    En general, para caulquier par de morfismos adjuntos $f_l\dashv f$, se cumple una desigualdad de la identidad de Frobenius.
    Sabemos que, por un lado, $b\wedge f(a)\leq b$ y entonces $f_l(b\wedge f(a))\leq f_l(b)$; y por otro lado,
    como $b\wedge f(a)\leq f(a)$, tenemos que $f_l(b\wedge f(a))\leq f_l(f(a))\leq a$; por lo tanto $f_l(b\wedge f(a))\leq f_l(b)\wedge a.$
\end{obs}

\subsection{Morfismos étales}

\begin{lema}
    Sea $f:A\to B$ un morfismo de marcos, $a\in A$, $b\in B$ y $g:\downarrow a\to \downarrow b$ otro morfismo de marcos tal que 
    el cuadrado \begin{tikzcd}
	A & B \\
	{\downarrow a} & {\downarrow b}
	\arrow["f", from=1-1, to=1-2]
	\arrow["{a\wedge-}"', from=1-1, to=2-1]
	\arrow["{b\wedge-}", from=1-2, to=2-2]
	\arrow["g"', from=2-1, to=2-2]
\end{tikzcd} conmuta, entonces $g(x)=b\wedge f(x)$ para todo $x\leq a$ y $g_*(y)=f_*(b\prec y)\wedge a$.
\end{lema}
\begin{proof}
    Sea $x\leq a$, entonces:
    $$g(x)=g(a\wedge x)=b\wedge f(x)$$
    ya que el diagrama conmuta. Por otro lado sea $y\leq b$, entonces:
    $$f_*(b\prec y)=f_*\circ (b\wedge-)_*(y)=(a\wedge-)_*\circ g_*(y)=a\prec g_*(y)$$
    ya que si el cuadrado conmuta, entonces conmuta el cuadrado de los respectivos adjuntos derechos.
    
    De lo anterior, por un lado tenemos que $f_*(b\prec y)\wedge a\leq g_*(y)$. Por otro lado, notemos que $g_*(y)\leq a$,
    y por lo obtenido previamente, tenemos que $b\wedge f(g_*(y))=g(g_*(y))\leq y$ (Esta última desigualdad porque $g\dashv g_*$).
    Luego eso es equivalente a que $f(g_*(y))\leq b\prec y$ (Por ser la implicación el adjunto derecho de sacar ínfimo), y por último
    $g_*(y)\leq f_*(b\prec y)$ (ya que $f\dashv f_*)$. Y por lo tanto $g_*(y)=f_*(b\prec y)\wedge a$.
\end{proof}

\section{Algunas equivalencias importantes}

\subsection{$\Et/A\cong\Gav(A)$}

\subsection{$\Con(A)\cong \Gav(A)$}

\subsection{Topos localico}
