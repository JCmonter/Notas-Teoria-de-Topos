\chapter{Gavillas sobre Marcos}

\section{Marcos y gavillas}

\subsection{Morfismos abiertos}

\begin{definition}[Morfismo abierto]
    Sea $f:A\to B$ un morfismo de marcos. Decimos que es abierto si existe $f_l:B\to A$, un morfismo de copos tal que:
    \begin{itemize}
        \item $f_l\dashv f$.
        \item Para cada $a\in A$ y $b\in B$ se tiene la identidad de Frobenius 
            $$f_l(b\wedge f(a))=f_l(b)\wedge a.$$
    \end{itemize}
\end{definition}

\begin{obs}
    En general, para caulquier par de morfismos adjuntos $f_l\dashv f$, se cumple una desigualdad de la identidad de Frobenius.
    Sabemos que, por un lado, $b\wedge f(a)\leq b$ y entonces $f_l(b\wedge f(a))\leq f_l(b)$; y por otro lado,
    como $b\wedge f(a)\leq f(a)$, tenemos que $f_l(b\wedge f(a))\leq f_l(f(a))\leq a$; por lo tanto $f_l(b\wedge f(a))\leq f_l(b)\wedge a.$
\end{obs}

\subsection{Morfismos étales}

\begin{lema}
    Sea $f:A\to B$ un morfismo de marcos, $a\in A$, $b\in B$ y $g:\downarrow a\to \downarrow b$ otro morfismo de marcos tal que 
    el cuadrado \begin{tikzcd}
	A & B \\
	{\downarrow a} & {\downarrow b}
	\arrow["f", from=1-1, to=1-2]
	\arrow["{a\wedge-}"', from=1-1, to=2-1]
	\arrow["{b\wedge-}", from=1-2, to=2-2]
	\arrow["g"', from=2-1, to=2-2]
\end{tikzcd} conmuta, entonces $g(x)=b\wedge f(x)$ para todo $x\leq a$ y $g_*(y)=f_*(b\prec y)\wedge a$.
\end{lema}
\begin{proof}
    Sea $x\leq a$, entonces:
    $$g(x)=g(a\wedge x)=b\wedge f(x)$$
    ya que el diagrama conmuta. Por otro lado sea $y\leq b$, entonces:
    $$f_*(b\prec y)=f_*\circ (b\wedge-)_*(y)=(a\wedge-)_*\circ g_*(y)=a\prec g_*(y)$$
    ya que si el cuadrado conmuta, entonces conmuta el cuadrado de los respectivos adjuntos derechos.
    
    De lo anterior, por un lado tenemos que $f_*(b\prec y)\wedge a\leq g_*(y)$. Por otro lado, notemos que $g_*(y)\leq a$,
    y por lo obtenido previamente, tenemos que $b\wedge f(g_*(y))=g(g_*(y))\leq y$ (Esta última desigualdad porque $g\dashv g_*$).
    Luego eso es equivalente a que $f(g_*(y))\leq b\prec y$ (Por ser la implicación el adjunto derecho de sacar ínfimo), y por último
    $g_*(y)\leq f_*(b\prec y)$ (ya que $f\dashv f_*)$. Y por lo tanto $g_*(y)=f_*(b\prec y)\wedge a$.
\end{proof}

\section{Algunas equivalencias importantes}

\subsection{$\Et/A\cong\Gav(A)$}

\subsection{$\Con(A)\cong \Gav(A)$}

\subsection{Topos localico}
